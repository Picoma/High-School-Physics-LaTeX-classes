\documentclass[tikz-cross]{HSP-Lecture}
\usepackage[utf8]{inputenc}

% Réglages :
%\definecolor{cPrim}{RGB}{15, 115, 0}
%\definecolor{cPrim}{RGB}{243,102,25} % Legrand Orange Book's "ocre"
\settoggle{versionProf}{false}

\chapNumber{N}
\title{Lecture title}

\begin{document}
	\maketitle
	
	\section{Section}
	\subsection{SubSection}
	In this chapter, we'll explore the capabilities of this specific \LaTeX{} class:
	
	\begin{itemize}
		\item \textbf{Cosmetic Layout :} This class focuses on enhancing the visual aesthetics of your document, ensuring a professional appearance.
		\item \textbf{Customizable Accent Color :} You can easily adjust the accent color using \\
		\verb|\definecolor{cPrim}{...}|.
		\item \textbf{Interactive Learning :} Certain sections can be hidden using the \verb|versionProf| switch in the preamble, serving as blank spaces for student engagement.
	\end{itemize}
	
	Now, let's examine some key elements:
	\begin{figure}[h]
		\centering
		\begin{tikzpicture}
			% Axe et lentille
			\draw[-stealth, thick] (-0.4\linewidth,0) -- (0.4\linewidth,0);
			\draw[stealth-stealth, thick] (0,-2) -- (0,2);
			
			% Centre optique :
			\draw (0,0) node[above right]{$O$};
			
			% Foyers
			\draw (-0.15\linewidth,0)
			node[cross=2.5pt,rotate=45] (F) {}
			node[below, yshift=-2pt]{$F$};
			
			\draw (0.15\linewidth,0)
			node[cross=2.5pt,rotate=45] (F') {}
			node[above, yshift=2pt]{$F'$};
			
			% Objet
			\draw[-stealth] (-0.3\linewidth,0)
			node (A) {}
			node[below] {A}
			-- ++(0,1)
			node (B) {}
			node[above] {B};
			
			% Image
			\draw[-stealth] (0.3\linewidth,0)
			node (A') {}
			node[above] (A') {A'}
			-- ++(0,-1)
			node (B') {}
			node[below] {B'};
			
			% Rayons
			\usetikzlibrary{decorations.markings}
			
			\begin{scope}[very thick,decoration={
					markings,
					mark=at position 0.125 with {\arrow{>}},
					mark=at position 0.375 with {\arrow{>}},
					mark=at position 0.625 with {\arrow{>}},
					mark=at position 0.875 with {\arrow{>}},}
				]
				\draw[thick, RoyalBlue, postaction={decorate}] (B.center) -- ++(0.3\linewidth,0) -- (F') -- (B'.center);
				\draw[thick, ForestGreen, postaction={decorate}] (B) -- ++(0,0) -- (B'.center);
				\draw[thick, BrickRed, postaction={decorate}] (B.center) -- (F) -- (0,-1) -- (B'.center);
			\end{scope}
			
			\draw[-stealth] (0.2\linewidth,1) node[anchor=south west]{+}-- ++(0.75,0);
			\draw[-stealth] (0.2\linewidth,1) -- ++(0,0.75);
		\end{tikzpicture}
		\caption{Lens, characterized by its optical center $O$ and foci $F$ and $F'$, forming the image $A'B'$ of an object $AB$.}
	\end{figure}
	
	\begin{coursBase}
		This section remains visible at all times, providing essential content.
	\end{coursBase}
	
	\begin{cours}{4cm}
		This section can be hidden using the \texttt{versionProf} switch, encouraging student interaction by allowing them to complete it.
	\end{cours}
	
	\begin{attention}
		This section contains critical information or reminders that should not be overlooked.
	\end{attention}

	\newpage
	\section*{Application exercise: the microscope}
	
	A microscope uses removable lenses to image a small, distant object on the retina of the eye, or a camera. 
	The system can be simply modeled\footnote{In reality, the complete system comprises at least three lenses: an objective, an eyepiece and the eye's crystalline lens. We're simplifying the situation here so that we can do the exercise...} as follows :
	
	\begin{center}
		\begin{tikzpicture}
			\usetikzlibrary{arrows.meta}
			\usetikzlibrary{decorations.markings}
			
			% Grille
			\draw [gray, step=1.0cm] (-7,-4) grid +(14,8);
%			\draw [very thin, gray, step=0.1cm,xshift=0cm, yshift=0cm] (0,0) grid +(17,8);
			
			% Lentille & axe optique
			\draw[very thick, {Stealth[width=3mm]}-{Stealth[width=3mm]}] (0,-3) -- (0,3);
			\draw[very thick, -stealth, dashed] (-7,0) -- (7,0);
			
			% Centre optique :
			\draw (0,0)
			node[cross=5pt, thick] {}
			node[above right]{$O$};
			
			% Foyers
			\draw (-2,0)
			node[cross=2.5pt,rotate=45] (F) {}
			node[below, yshift=-2pt]{$F$};
			
			\draw (2,0)
			node[cross=2.5pt,rotate=45] (F') {}
			node[above, yshift=2pt]{$F'$};
			
			% Objet
			\draw[-stealth] (-6,0)
			node (A) {}
			node[above] {A}
			-- ++(0,-2)
			node (B) {}
			node[below] {B};
			
			\iftoggle{versionProf}{
				% Image
				\draw[-stealth] (3,0)
				node (A') {}
				node[below] (A') {A'}
				-- ++(0,1)
				node (B') {}
				node[above] {B'};
				
				% Rayons
				\draw[thick, RoyalBlue] (B.center) -- ++(6,0) -- (B'.center);
				\draw[thick, ForestGreen] (B) -- ++(0,0) -- (B'.center);
				\draw[thick, BrickRed] (B.center) -- (0,1) -- (B'.center);
				
				% Légende
				\draw[thick, -stealth] (-6,2) -- ++(1,0) node[midway,below]{1mm};
				\draw[thick, -stealth] (-6,2) -- ++(0,1) node[midway,left]{1mm};
			}{}
		\end{tikzpicture}
	\end{center}
	
	\paragraph{Without completing the diagram:}
	\begin{enumerate}
		\item Give the values of $\overline{OA}$ and $\overline{OF'}$ and $\overline{AB}$.
		\item Determine by calculation the position of the image $A'B'$ of object $AB$ through the lens.
		\item Determine by calculation the direction and size of the image $A'B'$.
	\end{enumerate}
	
	\paragraph{By completing the diagram :}
	\begin{enumerate}
		\setcounter{enumi}{3}
		\item Draw the image $A'B'$ of object $AB$ through the lens. Check the results of the previous questions.
	\end{enumerate}

\end{document}