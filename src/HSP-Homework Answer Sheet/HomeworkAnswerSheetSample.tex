\documentclass[code]{HSP-AnswerSheet}
\usepackage{verbatim}

% Settings :
%\definecolor{cPrim}{RGB}{15, 115, 0}
\colorlet{cPrim}{Blue}

\title{Chapter ?? : Chapter title}
\date{Semaine n°1}

\begin{document}
	\maketitle
	
	\begin{multicols*}{2}
		
		\subsection*{Exercise 1 p.42 : Introduction}
		This class is purely cosmetic. It aims at automating some frequent tasks and giving a homogeneous appearance to all answer keys, in addition to automating some repetitive tasks (for example, changing the font for dyslexic students).
		
		\subsection*{Exercise 3 p.42 : Class Features}
		The (few) features of this class are :
		\begin{enumerate}[label=\bfseries\alph*.]
			\item The accent color can be changed using the \\ \verb|\colorlet{cPrim}{...}| command in the document preamble.
			\item The \verb*|enumitem| package is loaded to use the same numbering as that of the exercise (alphabetical, Arabic or Roman numeric numbering, ...)
			\item The \verb*|siunitx| and \verb*|cancel| packages are loaded by default :
			\begin{align*}
				v&=\frac{d}{\Delta t}\\
					\textbf{AN : }\quad
				v&=\frac{ \SI{3.0}{\meter} }{ \SI{5.0}{\second} }\\
				 &=\SI{0.6}{\meter\per\second}
			\end{align*}
		
			\item The curriculum requires that units are optional in the development of the calculation. To reflect this, the units can be grayed out using the \verb|\SIgray| command :
			\begin{align*}
				v&=\frac{d}{\Delta t}\\
				\textbf{AN : }\quad
				v&=\frac{ \SIgray{3.0}{\meter} }{ \SIgray{5.0}{\second} }\\
				&=\SI{0.6}{\meter\per\second}
			\end{align*}
		\end{enumerate}
		
		\subsection*{Exercise 3 p.42 : Class Options}
		Some options can be provided to the class :
		\begin{description}
			\item[code] Imports the \verb|listings| package to typeset code (as Python for physics is part of the curriculum) :
			\begin{lstlisting}[language=Python]
#!/usr/bin/python3
def print_hello_world():
	print("Hello World !")

print_hello_world()
			\end{lstlisting}
		\item[OpenDyslexic] Changes the document font to OpenDyslexic, to accommodate some dyslexic students.
		
		{\fontspec{OpenDyslexic} Here is a sample of the font.}
		
		\item[doublespacing] Changes the font spacing to 2, to accommodate some dyslexic students.
		\end{description}
		
	\end{multicols*}
\end{document}