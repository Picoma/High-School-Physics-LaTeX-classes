\documentclass[12pt,code]{HSP-Test}

%%%%%%%%%%%%%%%%%%%%%%%%%%%%% Class settings : %%%%%%%%%%%%%%%%%%%%%%%%%%%%%

% Primary color :
\colorlet{cPrim}{Mahogany} % To use a predefined color, i.e. from dvipsnames
%\definecolor{cPrim}{RGB}{15, 115, 0} % To define your own color

% Headers & footers : test's metadata
\title{Test n°?}
\duration{1 hour}
\class{CLASS}
\date{Week n°??}

% Versionning : in case some students are absent on the scheduled test day. Purely cosmetic.
\settoggle{withVersions}{false}
\version{A}

% Switch between the test subject and the answer sheet
\settoggle{answerSheet}{true}

%%%%%%%%%%%%%%%%%%%%%%%%%% Additionnal settings : %%%%%%%%%%%%%%%%%%%%%%%%%%

\graphicspath{{Figures/}}

%%%%%%%%%%%%%%%%%%%%%%%%%%%%%%%%%%%%%%%%%%%%%%%%%%%%%%%%%%%%%%%%%%%%%%%%%%%%

\begin{document}
	\maketitle{% The text below will not appear on the answer sheet.
		\begin{center}\sffamily
			This text will only appear on the answer sheet.
			
			Useful to remind students to be careful about their writing, ...
		\end{center}
	}
	
	Introduction to the theme of the test : you can add text, images, quotes, etc. The current "test" gives an overlook over the functionalities provided.
	
	\section{Getting started -- changing the formatting}
	
	As you can see, exercises are defined using the "section" command; one exercise.
	
	In practice, this paragraph could provide the context of a problem that will be addressed in several sub-parts (defined by the \texttt{subsection} command).
	
	\subsection{Adaptive document structure}
	One of the main features of this class is the use of conditionals to modify the document's structure.
	
	\begin{enumerate}[label=\bfseries\arabic*.]
		\item To change the entire document and see the results for those questions, set the \texttt{answerSheet} toggle to \texttt{true} (located in the preamble of the \texttt{.tex} document, line 20).
		\begin{answer}
			This is how you do it! Notice that:
			\begin{itemize}
				\item A margin appears on the right where the questions' points are displayed.
				\item The reminder paragraph below the title disappears.
				\item The title now includes "Answer sheet."
			\end{itemize}
			
			All of this can be achieved with the flip of a switch!
		\end{answer}
	
		\item How are points indicated when in \texttt{answerSheet} mode?
		\begin{answer}
			Points for a given question are assigned using the \texttt{pts} command. For example, by using \Verb{\pts{0.5}} at the end of this paragraph, the value of this item is displayed in the right margin. \pts{0.5}
			
			You can use this command multiple times in a single answer if the answer requires multiple parts graded individually. \pts{0.5}
			
			Please note that points only appear in \texttt{answerSheet} mode.
		\end{answer}
	
		\item Can you guess how I integrated this class into my workflow to be efficient ?
		\begin{answer}
			My idea was to :
			\begin{itemize}
				\item Write a draft test subject in this document without the answers.
				\item Quickly try it out on paper to prevent any major oversights in the test's design.
				\item Switch to \texttt{answerSheet} mode.
				\item Correct typos in the questions \emph{while} coding\footnote{Coding, not typing !} the answer sheet.
			\end{itemize}
			
			That way, I can design a test \emph{only by writing its answer sheet}.
			
			This allows me to design a test by \emph{only writing its answer sheet}, eliminating the risk of pasting content in the wrong document and ensuring no missing answer key. 
			
			Additionally, the entire document is automatically formatted, saving significant time.
			
			To minimize the risk of errors, values are coded rather than typed, with the computer computing the results and displaying them. See section \ref{sec:PythonTex} for more details.
		\end{answer}
	\end{enumerate}
	
	\subsection{Other cosmetic settings}
	This section outlines various macros, environments, and pre-loaded packages designed to enhance the cosmetic aspects of your document. These elements are provided for convenience and serve as wrappers around specific packages:
	
	\begin{doc}[Cosmetic environments and macros]
		\setlength{\parindent}{0pt}
		
		\textbf{Preamble macros :}	% Despite the classe's pros,
		\vspace{-\baselineskip}		% sometimes hacks are still necessary...
		\begin{multicols}{2}
			\begin{itemize}
				\item \Verb{\title{Test n°?}} : Sets the title at the top of the sheet \textbf{and} in the top right corner of every page.
				\item \Verb{\duration{1 hour}} : Displays small text below the title, but it only appears on the test subject.
				\item \Verb{\class{CLASS}} : Places text on the top left corner of the page.
				\item \Verb{\date{Week n°??}} : Includes text on the bottom left corner of the page.
				\item \Verb{\colorlet{cPrim}{...}} and \newline
				\Verb{\definecolor{cPrim}{...}} : Allows you to change the accent color. This is particularly useful for organizing content by grade level and for aesthetic purposes.
			\end{itemize}
		\end{multicols}
		
		\textbf{Provided environments :}
		\vspace{-\baselineskip}
		\begin{multicols}{2}
			\begin{itemize}
				\item \Verb{\begin{doc}[Additionnal title]} : This creates a framed document for students to analyze, similar to the one you're currently reading. It's essentially an \texttt{mdframed} environment.
				\item \Verb{\begin{answer}} : This environment is an \texttt{mdframed} wrapper around an \texttt{environ} custom environment that's displayed exclusively in the answer sheet.
				\item \texttt{enumerate}, as modified by the \texttt{enumitem} package : This enhances enumerated lists by allowing enumeration to be interrupted and resumed using the optional argument \texttt{resume}. It also provides modifications for item labels.
			\end{itemize}
		\end{multicols}
		
		\textbf{Pre-loaded packages :}
		\vspace{-\baselineskip}
		\begin{multicols*}{2}
			\begin{itemize}
				\item \texttt{multicols}
				\item \texttt{TiKZ}
				\item \texttt{mdframed}
				\item \texttt{enumitem}
				\item \texttt{xcolor} with \texttt{dvipsnames}
				\item \texttt{etoolbox}
				\item \texttt{siunitx}
			\end{itemize}
		\end{multicols*}
	\end{doc}
	
	\begin{enumerate}[label=\bfseries\arabic*.]
		\item Since the \texttt{mdframed} package is preloaded, you can create custom frames, as demonstrated below:
	\end{enumerate}
	
	\begin{mdframed}[frametitle={\colorbox{white}{\space Data sheet\space}},
					 frametitleaboveskip=-\ht\strutbox]
		Data sheets are essential in physics. We often deal with values and formulas, but memorizing them is not always necessary.
		\begin{itemize}
			\item $v = \frac{d}{\Delta t}$
			\item The speed of light is $\SI{3.00e8}{\meter\per\second}$. Note the use of \texttt{siunitx} for formatting the value !
		\end{itemize}
	\end{mdframed}
	
	\begin{mdframed}[frametitle={\colorbox{cPrim!10}{\space Warning ! \space}},
					 frametitleaboveskip=-\ht\strutbox,
					 backgroundcolor=cPrim!10,
					 linecolor=cPrim,
%					 startinnercode={\hspace{\parindent}}
					 ]
		This information is crucial; students should not overlook it.
	\end{mdframed}
	
	\begin{enumerate}[resume,label=\bfseries\arabic*.]
		\item Notice how, despite exiting the previous \texttt{enumerate} environment, the numbering resumes. This is made possible by the \texttt{resume} argument provided by the \texttt{enumitem} package.
	\end{enumerate}

	\section{PythonTeX and its wrappers}
	\label{sec:PythonTex}
	\subsection{Context}
	\label{subsec:pythonTex_context}
	The second significant feature of this class is its \emph{ability to compute answers}, rather than manually type them.
	
	In high school physics, calculations often follow a consistent structure:
	\begin{itemize}
		\item Write down the formula ;
		\item Replace the terms with their values (and optionally their unit) ;
		\item Compute the final value and the final unit.
	\end{itemize}
	
	For example, a student might write the following on their paper:
	\begin{align*}
		v&=\frac{d}{\Delta t}\\
		 &=\frac{\SI{30}{\meter}}{\SI{4.0}{\second}}\\
		 &=\SI{7.5}{\meter\per\second}
	\end{align*}
	
	Manually copying this process into an answer sheet multiple times can lead to errors. Common mistakes include:
	\begin{itemize}
		\item Changing the value in the second step but not in the third.
		\item Forgetting to convert units between the second and third step (if an intermediate unit conversion is added).
		\item Changing the units of quantities in the question but forgetting to update them in the answer.
		\item Inconsistencies with significant figures, and more.
	\end{itemize}
	In summary, this is an almost automatic but error-prone process when done by hand. Why not automate it?
	
	\subsection{Python, Python\TeX}
	\subsubsection{Why Python in \LaTeX{} ?}
	To accomplish this, I decided to integrate Python into the \TeX{} document using the PythonTeX library. I prefer this approach over a purely \LaTeX{}-based solution for several reasons:
	\begin{itemize}
		\item I'm not comfortable with programming in \LaTeX{}; I might have managed automatic value calculations, but certainly not unit conversions.
		I discovered a Python library, \href{https://github.com/birkenfeld/ipython-physics}{\texttt{physics.py}, authored by Georg Brandl}, that already performed exactly what I needed. It was much easier to adapt this library to generate \LaTeX{} code than to start from scratch.
		\item Programming in Python for physics aligns with the curriculum, so I already had a working Python distribution installed.
	\end{itemize}
	
	\subsubsection{Quick example}
	To reuse the example from subsection \ref{subsec:pythonTex_context}, follow these steps (refer to the \texttt{.tex} document for the full process):
	\begin{enumerate}
		\item Define the quantities using a \texttt{pycode} block ;
		% Mind that the following code will be executed as is : don't indent it, or it will cause an IndentationError !
		\begin{pycode}
d  = Q("30.0 m", prec=3)
Δt = Q("4.0  s", prec=2)
		\end{pycode}
		
		\item Type the formula : 
			$$v=\frac{d}{\Delta t}$$
		\item Re-enter the same formula, replacing every symbol whose value you want with the corresponding Python variable, encapsulated within the \Verb{\Py} macro :
			$$v=\frac{\Py{d}}{\Py{Δt}}$$
		\item Re-enter the same formula, \emph{this time within the \Verb{\Py} macro, as if it were a Python operation} :
			$$v=\Py{d/Δt}$$
		\item Compile the document following the Python\TeX{} documentation :\\
		Xe\LaTeX{} $\rightarrow$ PythonTex $\rightarrow$ Xe\LaTeX{} again.
	\end{enumerate}
	
	Now, try changing the values in the \texttt{pycode} block above and recompile the document (step 5). You will see that the values are automatically updated.
	
	Note that, while the result is as expected, the significant digits of $\Delta t$ are not correctly displayed. This issue stems from the way Python formats strings and currently has no solution other than a workaround (explained in doc. \ref{doc:python_commands}), which involves manually adding the significant zero using the optional argument with \Verb{\Py} :
		$$v=\frac{\Py{d}}{\Py[.0]{Δt}}$$
	
	\subsubsection{General usage}
	Below is a list of commands and macros that can be used in a document:
	\begin{doc}[\LaTeX{} macros]
		\begin{description}[style=nextline, format=\bfseries]
			\item[\Verb{\Py[additionnal]{PhysicalQuantity}}]
				
				A wrapper around the Python\TeX{} \Verb{\pyc} command, which calls the Python \texttt{PhysicalQuantity.latex} method. It prints the string \Verb{\SI{value+additional}{unit}} and renders it using \LaTeX.
				
				The \texttt{additional} parameter can be used to manually add missing significant digits or uncertainties (which are not automatically computed).
			\item[\Verb{\PyGray[additionnal]{PhysicalQuantity}}]
				This macro is identical to \Verb{\Py}, except that it displays the unit in gray. According to the curriculum, writing the unit during calculations is optional. Graying out the unit helps convey this idea while reminding students of the unit's importance (and the existence of units at all...).
		\end{description}
	\end{doc}
	
	\begin{doc}[Python commands (to use within a \texttt{pyblock})]
		\label{doc:python_commands}
		\begin{description}[style=nextline, format=\bfseries]
			\item[\Verb{Q(value, unit=None, stdev=None, prec=3, force_scientific=False)}]
				The \texttt{PhysicalQuantity} (aliased as \texttt{Q}) object, with two constructor calling patterns:
				\begin{enumerate}
					\item \texttt{PhysicalQuantity(value, unit)}: Here, \texttt{value} can be any number, and \texttt{unit} is a string defining the unit.
					\item \texttt{PhysicalQuantity(value\_with\_unit)}: In this form, \texttt{value\_with\_unit} is a string containing both the value and the unit, e.g., \texttt{'1.5 m/s'}. This format is provided for more convenient interactive use.
				\end{enumerate}
				\begin{description}[format=\ttfamily]
					\item[stdev]\label{stdev} (Integer): Standard deviation or uncertainty. Currently, this is cosmetic only and is not automatically computed, unlike units.
					\begin{pyconsole}
from physics import Q
print(Q("150 m/s").latex())
print(Q("150 m/s", stdev=0.4).latex())
					\end{pyconsole}
					
					\item[prec] (Integer) : Number of digits to display. It uses Python's general (\texttt{g}) string formatting method to display those significant digits.
					\begin{pyconsole}
print(Q("150 m/s").latex())         # Default : 3 digits
print(Q("150 m/s", prec=5).latex()) # 5 digits
					\end{pyconsole}
					
					\textcolor{cPrim}{\bfseries Warning :} due to string formatting, if the value is too close to 1, scientific notation may not be used, and significant digits will be ignored. Example :
					\begin{pyconsole}
print(Q("1.5 m/s").latex())         # Expected value : 1.50
print(Q("1.5 m/s", prec=5).latex()) # Expected value : 1.5000
					\end{pyconsole}
					\textbf{Workaround :} Additional 0s can be added manually from \LaTeX{} for now, using \Verb{\Py}'s optional argument. To keep the same example :
					
					\begin{tabular}{rl}
						\Verb{\Py{Q("1.5 m/s")}}    : & \Py{Q("1.5 m/s")}\\
						\Verb{\Py[0]{Q("1.5 m/s")}} : & \Py[0]{Q("1.5 m/s")}\\
						\Verb{\Py[000]{Q("1.5 m/s")}} : & \Py[000]{Q("1.5 m/s")}
					\end{tabular}
					
					\item[\Verb{force_scientific}] (Boolean) : Forces the use of scientific notation (\emph{e}) in string formatting to display quantity values.
					\begin{pyconsole}
print(Q("15 m/s").latex())
print(Q("15 m/s", force_scientific=True).latex())
					\end{pyconsole}
				\end{description}
		\end{description}
	\end{doc}
\end{document}