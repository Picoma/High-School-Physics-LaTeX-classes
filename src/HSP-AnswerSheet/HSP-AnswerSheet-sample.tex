\documentclass[code]{HSP-AnswerSheet}
\usepackage{verbatim}

% Settings :
%\definecolor{cPrim}{RGB}{15, 115, 0}
\colorlet{cPrim}{Blue}

\title{Chapter ?? : Chapter title}
\date{Week n°1}

\begin{document}
	\maketitle
	
	\begin{multicols*}{2}
		
		\subsection*{Exercise 1 p.42 : Introduction}
		This class serves a purely cosmetic purpose. It aims to automate common tasks and maintain a consistent appearance for all answer keys. Additionally, it streamlines repetitive tasks, such as changing fonts for students with dyslexia.
		
		\subsection*{Exercise 2 p.42 : Class Features}
		The class offers a few key features :
		\begin{enumerate}[label=\bfseries\alph*.]
			\item You can change the accent color using the \\ \verb|\colorlet{cPrim}{...}| command in the document preamble.
			\item The class loads the \verb|enumitem| package to provide flexible and consistent numbering options, including alphabetical, Arabic, or Roman numeric numbering.
			\item The class comes with the \verb|siunitx| and \verb|cancel| packages preloaded, allowing you to format units and expressions effortlessly. For example:
			\begin{align*}
				v&=\frac{d}{\Delta t}\\
					\textbf{AN : }\quad
				v&=\frac{ \SI{3.0}{\meter} }{ \SI{5.0}{\second} }\\
				 &=\SI{0.6}{\meter\per\second}
			\end{align*}
		
			\item To comply with the curriculum that allows units to be optional in calculation development, you can use the \verb|\SIgray| command to gray out units:
			\begin{align*}
				v&=\frac{d}{\Delta t}\\
				\textbf{AN : }\quad
				v&=\frac{ \SIgray{3.0}{\meter} }{ \SIgray{5.0}{\second} }\\
				&=\SI{0.6}{\meter\per\second}
			\end{align*}
		\end{enumerate}
		
		\subsection*{Exercise 3 p.42 : Class Options}
		Some options can be provided to the class :
		\begin{description}
			\item[code] Enables code typesetting by importing the \verb|listings| package. This feature is useful for teaching Python for physics :
			\begin{lstlisting}[language=Python]
#!/usr/bin/python3
def print_hello_world():
	print("Hello World !")

print_hello_world()
			\end{lstlisting}
		\item[OpenDyslexic] Changes the document font to OpenDyslexic, making it more accessible for dyslexic students.
		
		{\fontspec{OpenDyslexic} Here is a sample of the font.}
		
		\item[doublespacing] Adjusts the font spacing to 2, which is helpful for students with dyslexia.
		\end{description}
		
	\end{multicols*}
\end{document}